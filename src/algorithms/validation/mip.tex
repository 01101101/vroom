\documentclass{article}
\usepackage{nicematrix}
\usepackage{fullpage}

\begin{document}

\section{Mathematical model}

Let $R=\{T_0, T_1, \ldots, T_n, T_{n + 1}\}$ be the fixed order of
tasks within the route for a given vehicle. The vehicle has to perform
$n$ geolocated tasks ($T_1$ to $T_n$) while $T_0$ and $T_{n + 1}$ are
its optional start and end tasks. For all $i$, we denote $s_i$ the
service time for task $i$, $d_i$ the travel duration from task $T_i$
to the next non-break task. Task $T_i$ ($1\leq i \leq n$) has $K_i$
time windows $[a_{i1}, b_{i1}]$, $\dots$, $[a_{iK_i}, b_{iK_i}]$ for
expected task start. In particular for vehicle start we have $s_0 = 0$
while working hours associated to the vehicle are $[a, b]$.\\

The following program aim at deciding start times $t_i$ for all tasks
$T_i$ in order to minimize timing violations measured by variable
$Y_i$: either \textit{lead time} whenever a task starts earlier than
it's available date, either \textit{delay} when a task starts later
that it's deadline. For each task $T_i$ ($1\leq i\leq n$), the chosen
time window is decided by binary variables $X_{ik}$
($0\leq k \leq K_i$). Variables $\delta_i$ allow to split required
travel duration before and after a (set of) breaks. Let $J$ be the
subset of indices $i$ ($0\leq i \leq n$) such that $T_i$ is not a
break. For all $i\in J$, let $B_i$ be the number of tasks following
task $T_i$ that are breaks.

\begin{quote}
  Minimize
  \begin{align}
    M_1 \times \sum_{i = 0}^{n + 1} Y_i + M_2 \times (t_{n +1} - t_0) + M_3\times\sum_{i = 1}^{n}t_i + \sum_{i\in J}\sum_{k = i}^{i + B_i}(k - i)\times\delta_k\label{obj}\tag{Obj}
  \end{align}
  Subject to
  \begin{align}
    t_{i + 1} - t_i - \delta_i &\geq s_i && 0 \leq i \leq n\label{precedence}\tag{$P_i$}\\
    Y_0 + t_0 &\geq a &&  \label{lead_start}\tag{$L_0$}\\
    Y_i + t_i - \sum_{k = 0}^{K_i}a_{ik}X_{ik}&\geq 0 && 1 \leq i \leq n \label{lead}\tag{$L_i$}\\
    t_i - Y_i - \sum_{k = 0}^{K_i}b_{ik}X_{ik} &\leq 0 && 1 \leq i \leq n \label{delay}\tag{$D_i$}\\
    t_{n + 1} - Y_{n + 1} &\leq b && \label{delay_end}\tag{$D_{n + 1}$}\\
    \sum_{k = 0}^{K_i}X_{ik} &= 1 && 1 \leq i \leq n \label{tw_sum}\tag{$S_i$}\\
    \sum_{k = i}^{i + B_i}\delta_k &= d_i && \forall i\in J \label{break_duration}\tag{$\Delta_i$}\\
    t_i &\geq 0 && 0 \leq i \leq n + 1 \nonumber\\
    Y_i &\geq 0 && 0 \leq i \leq n + 1 \nonumber\\
    \delta_i &\geq 0 && 0 \leq i \leq n \nonumber\\
    X_i &\in \{0; 1\} && 1 \leq i \leq n \nonumber
  \end{align}
\end{quote}

The optimization objective (\ref{obj}) uses values
$M_1 >> M_2 >> M_3 >> 1$ to combine the sum of time violations as
first objective, then route makespan as a secondary objective, then
tasks start date, and last force split travel time to be done as soon
as possible (required for situations with breaks surrounded by waiting
time). Precedence constraints (\ref{precedence}) make sure service and
travel times are accounted for between task start times. In the time
violation constraints for lead time (\ref{lead}) and delay
(\ref{delay}), variable $Y_i$ is zero iff task start matches the
expected time window. Sums (\ref{tw_sum}) ensure exactly one time
window is picked for each task. Sums (\ref{break_duration}) ensure the
required travel duration between any two consecutive non-break tasks
is split across all intermediate breaks. In particular if $T_i$ and
$T_{i + 1}$ are both non-break tasks, we have $\delta_i = d_i$. In
practice, variables $t_i$ can be bounded by additional user-defined
hard constraints.

\section{glpk matrix layout}

\setcounter{MaxMatrixCols}{30}

\begin{center}
  \rotatebox[origin=c]{90}{
    $\begin{NiceMatrix}
      & t_0 & t_1 & \Cdots & t_n & t_{n + 1} & Y_0 & Y_1 & \Cdots & Y_n & Y_{n + 1} & X_{10} & \Cdots & X_{1K_1} & \Cdots & X_{n0} & \Cdots & X_{nK_n} & \delta_0 & & \Cdots & \delta_n\\
      P_0 & -1 & 1 & & & & & & & & & & & & & & & & -1 \\
      \Vdots & & \Ddots & \Ddots & & & & & & & & & & & & & & & & \Ddots \\
      \Vdots & & &  \Ddots & \Ddots & & & & & & & & & & & & & & & & \Ddots \\
      P_n & & & & -1 & 1 & & & & & & & & & & & & & & & & -1 \\\hline
      L_0 & 1 & & & & & 1 \\
      L_1 & & 1 & & & & & 1 & & & & -a_{10} & \Cdots & -a_{1K_1} \\
      \Vdots & & & \Ddots & & & & & \Ddots & & & & & & \Ddots \\
      L_n & & & & 1 & & & & & 1 & & & & & & -a_{n0} & \Cdots & -a_{nK_n} \\\hline
      D_1 & & 1 & & & & & -1  & & & & -b_{10} & \Cdots & -b_{1K_1} \\
      \Vdots & & & \Ddots & & & & & \Ddots & & & & & & \Ddots \\
      D_n & & & & 1 & & & & & -1 & & & & & & -b_{n0} & \Cdots & -b_{nK_n} \\
      D_{n + 1} & & & & & 1 & & & & & -1\\\hline
      S_1 & & & & & & &  & & & & 1 & \Cdots & 1 \\
      \Vdots & & & & & & & & & & & & & & \Ddots \\
      S_n & & & & & & & & & & & & & & & 1 & \Cdots & 1 \\\hline
      \Delta_0 & & & & & & & & & & & & & & & & & & 1\cdots \\
      \Vdots & & & & & & & & & & & & & & & & & & & \Ddots \\
      \Vdots & & & & & & & & & & & & & & & & & & & & \Ddots \\
      \Delta_{max(J)} & & & & & & & & & & & & & & & & & & & & & \cdots 1 \\\hline
      Makespan & -1 & & & & 1\\
    \end{NiceMatrix}$
  }
\end{center}

\end{document}
